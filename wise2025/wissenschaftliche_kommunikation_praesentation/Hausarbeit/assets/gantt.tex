\begin{figure}[htbp]
    \centering
    \begin{ganttchart}[
        expand chart=\textwidth,
        vgrid, hgrid,
        y unit chart=0.7cm,
        bar height=0.4,
        bar top shift=0.3,
        title height=1,
        title label font=\bfseries\small,
        bar label font=\small,
        group label font=\small\bfseries,
        milestone label font=\small\itshape
    ]{1}{12}

        % Titelzeile
        \gantttitle{Zeitplan in Wochen (12 Wochen Gesamt)}{12} \\
        \gantttitlelist{1,...,12}{1} \\

        % --- Phase 1 ---
        \ganttgroup{1. Analyse \& Theorie}{1}{4} \\
        \ganttbar{Recherche \& Framework-Analyse}{1}{2} \\
        \ganttbar{Verfassen: Stand der Forschung}{2}{4} \\

        % --- Phase 2 ---
        \ganttgroup{2. Implementierung}{3}{8} \\
        \ganttbar{Electron-Prototyp (Baseline)}{3}{4} \\
        \ganttbar{Tauri \& NSIS Konfiguration}{5}{7} \\
        \ganttbar{Durchführung Messreihen}{8}{8} \\

        % --- Phase 3 ---
        \ganttgroup{3. Finalisierung}{9}{12} \\
        \ganttbar{Auswertung \& Diskussion}{9}{10} \\
        \ganttbar{Feinschliff \& Abstract}{10}{11} \\
        \ganttbar{Puffer \& Abgabe}{12}{12} \\

        \ganttmilestone{Abgabe}{12}

        % Verbindungen (Links) visualisieren Abhängigkeiten
        \ganttlink{elem1}{elem2} % Recherche -> Schreiben
        \ganttlink{elem3}{elem4} % Schreiben -> Electron Start
        \ganttlink{elem4}{elem5} % Electron -> Tauri
        \ganttlink{elem5}{elem6} % Tauri -> Messen
        \ganttlink{elem6}{elem7} % Messen -> Auswerten

    \end{ganttchart}
    \caption{Zeitplan der Bachelorarbeit: Paralleles Verfassen der theoretischen Grundlagen während der Implementierungsphase.}
    \label{fig:zeitplan}
\end{figure}
