% Dokumentation für das Paket
% https://ctan.ebinger.cc/tex-archive/graphics/pgf/contrib/pgfgantt/pgfgantt-doc.pdf
\begin{ganttchart}[
    vgrid, hgrid,
    y unit title=0.6cm,
    x unit=0.9cm, % Etwas schmaler, damit es auf die Seite passt
    title height=1,
% Styling
    bar label font=\footnotesize,
    group label font=\footnotesize\bfseries,
    milestone label font=\footnotesize\itshape,
]{1}{12}

    % Titelzeile: Wochen
    \gantttitle{Zeitplan in Wochen}{12} \\
    \gantttitlelist{1,...,12}{1} \\

    % Phase 1: Vorbereitung
    \ganttgroup{1. Analyse \& Konzept}{1}{3} \\
    \ganttbar{Literaturrecherche (ISO, Frameworks)}{1}{2} \\
    \ganttbar{Detailkonzept Messaufbau}{2}{3} \\

    % Phase 2: Praxis
    \ganttgroup{2. Implementierung \& Test}{3}{7} \\
    \ganttbar{Electron-Prototyp (Baseline)}{3}{4} \\
    \ganttbar{NSIS-Installer Konfiguration (Tauri)}{4}{6} \\
    \ganttbar{Durchführung Messreihen (CPU/RAM)}{6}{7} \\

    % Phase 3: Schreiben
    \ganttgroup{3. Ausarbeitung}{7}{12} \\
    \ganttbar{Auswertung der Ergebnisse}{7}{8} \\
    \ganttbar{Verfassen der Kapitel}{8}{11} \\
    \ganttbar{Korrekturlesen \& Formatierung}{11}{12} \\

    % Meilenstein
    \ganttmilestone{Abgabe}{12}

    % Verknüpfungen
    \ganttlink{elem1}{elem2} % Literatur -> Konzept
    \ganttlink{elem2}{elem4} % Konzept -> Prototyp
    \ganttlink{elem6}{elem7} % Messung -> Auswertung

\end{ganttchart}
