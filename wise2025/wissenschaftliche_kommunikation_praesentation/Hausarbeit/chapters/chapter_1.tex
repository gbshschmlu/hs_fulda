% nutzen Sie \chapter wenn Dokumentenklasse auf "book" ist
% ansonsten nutzen Sie \section bei articel und anderen
%\chapter{Einleitung}
\section{Einleitung und Motivation} \label{sec:einleitung}


\subsection{Problemstellung} \label{sec:problemstellung}
Bei der Zusammenarbeit mit externen Zulieferern ist eine präzise Übermittlung von Produktspezifikationen (PSPs) essenziell.
Das Tool \textit{GBI} standardisiert diesen Prozess: Lieferanten können Formulare ausfüllen, technische Zeichnungen synchronisieren und diese in validiertem JSON-Format als ZIP-Archiv per E-Mail versenden.

Die technische Verteilung dieser Software unterliegt strengen Restriktionen:
\begin{enumerate}
    \item \textbf{Heterogene Hardware:} Die Ausstattung der Lieferanten ist unbekannt.
    Teilweise kommen ältere Systeme (Windows 10) mit begrenzten Ressourcen zum Einsatz.
    \item \textbf{Verteilung und Bandbreite:} Bei Verteilung über E-Mail oder in Regionen mit limitierter Internetbandbreite ist die Dateigröße kritisch.
    \item \textbf{Installationsberechtigungen:} Viele Lieferanten verfügen über keine Administratorrechte.
    Die Installation muss im \("\)User Mode\("\) durchführbar sein und unternehmensspezifische Anpassungen (Lizenztexte, Branding) unterstützen.
\end{enumerate}

Klassische Ansätze wie Electron lösen das Cross-Platform-Problem durch Bündelung einer Chromium-Instanz, was jedoch zu Installer-Größen von über 80 MB und hohem Speicherverbrauch führt \cite{thangaduraiElectronVsWeb2024}.


\subsection{Motivation} \label{sec:motivation}
Tauri v2 verspricht durch Trennung von Frontend (Web-Technologien) und Backend (Rust) sowie Nutzung der systemeigenen Webview (WebView2 unter Windows \cite{webview2Docs}) eine signifikante Reduktion von Artefaktgröße und Ressourcenverbrauch.
\begin{figure}[htbp]
    \centering
    \includegraphics[width=1.0\textwidth]{assets/img/architektur_vergleich.png}
    \caption[Architekturvergleich Electron vs. Tauri]{Architekturvergleich: Electron bündelt die Browser-Engine (Chromium), während Tauri auf die systemseitige Webview zugreift.}
    \source{Eigene Darstellung in Anlehnung an \cite{tauriDocs2025}}
    \label{fig:architektur}
\end{figure}
Zudem bietet Tauri v2 eine tiefe Integration des \textit{Nullsoft Scriptable Install System} (NSIS).
Dies ermöglicht maßgeschneiderte Setup-Routinen ohne Admin-Rechte – ein Feature, das mit alternativen Installern (z.B. WiX) komplex umsetzbar ist.
Diese Arbeit überprüft, ob diese technologischen Vorteile die strengen Anforderungen des Anwendungsfalls erfüllen.
