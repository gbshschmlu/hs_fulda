% nutzen Sie \chapter wenn Dokumentenklasse auf "book" ist
% ansonsten nutzen Sie \section bei articel und anderen
%\chapter{Einleitung}
\section{Einleitung und Motivation} \label{sec:einleitung}


\subsection{Problemstellung} \label{sec:problemstellung}
Im Rahmen der Zusammenarbeit mit externen Zulieferern ist eine präzise Übermittlung von Produktspezifikationen (PSPs) essenziell.
Zur Standardisierung dieses Prozesses wird das Tool \textit{GBI} entwickelt, welches Lieferanten ermöglicht, komplexe Formulare auszufüllen, technische Zeichnungen zu synchronisieren und diese in einem validierten JSON-Format gebündelt als ZIP-Archiv per E-Mail zu versenden.

Die technische Verteilung dieser Software unterliegt jedoch strengen Restriktionen:
\begin{enumerate}
    \item \textbf{Heterogene Hardware:} Die Ausstattung der Lieferanten ist dem Unternehmen nicht im Detail bekannt.
    Es muss davon ausgegangen werden, dass teilweise ältere Systeme (Windows 10) mit begrenzten Arbeitsspeicherressourcen zum Einsatz kommen.
    \item \textbf{Verteilung und Bandbreite:} Da die Software teilweise über E-Mail-Verteiler oder in Regionen mit limitierter Internetbandbreite bereitgestellt wird, ist die Dateigröße des Installers ein kritischer Faktor.
    \item \textbf{Installationsberechtigungen:} Viele Lieferanten verfügen auf ihren Firmenrechnern über keine Administratorrechte.
    Der Installationsprozess muss daher ohne Erhöhung der Privilegien ("User Mode") durchführbar sein und gleichzeitig unternehmensspezifische Anpassungen (Lizenztexte, Branding) unterstützen.
\end{enumerate}

Klassische Ansätze wie Electron lösen das Cross-Platform-Problem durch das Bündeln einer Chromium-Instanz, was jedoch typischerweise zu Installer-Größen von über 80 MB und hohem Speicherverbrauch führt \cite{thangaduraiElectronVsWeb2024}.


\subsection{Motivation} \label{sec:motivation}
Das Framework Tauri v2 verspricht durch die Trennung von Frontend (Web-Technologien) und Backend (Rust) sowie die Nutzung der im Betriebssystem vorhandenen Webview (WebView2 unter Windows) eine signifikante Reduktion der Artefaktgröße und des Ressourcenverbrauchs.
\begin{figure}[htbp]
    \centering
    \includegraphics[width=1.0\textwidth]{assets/img/architektur_vergleich.png}
    \caption{Architekturvergleich: Electron bündelt die Browser-Engine (Chromium), während Tauri auf die systemseitige Webview zugreift.}
    \source{Eigene Darstellung in Anlehnung an \cite{tauriDocs2025}}
    \label{fig:architektur}
\end{figure}
Zudem bietet Tauri v2 eine tiefe Integration des \textit{Nullsoft Scriptable Install System} (NSIS).
Dies könnte die Erstellung maßgeschneiderter Setup-Routinen ermöglichen, die ohne Admin-Rechte funktionieren – ein Feature, das mit alternativen Installern (z.B. WiX) nur komplex umsetzbar ist.
Die Motivation dieser Arbeit liegt in der wissenschaftlichen Überprüfung, ob diese technologischen Vorteile in der Praxis bestand haben und die strengen Anforderungen des Anwendungsfalls erfüllen.
