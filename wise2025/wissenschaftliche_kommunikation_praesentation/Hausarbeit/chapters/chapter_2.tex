% nutzen Sie \chapter wenn Dokumentenklasse auf "book" ist
% ansonsten nutzen Sie \section bei articel und anderen
%\chapter{Kapitel} \label{kapitel}
\section{Zielsetzung und Forschungsfragen} \label{sec:zielsetzung}

Ziel der geplanten Abschlussarbeit ist die technische Evaluierung von Tauri v2 als Framework für industrielle Desktop-Anwendungen mit Fokus auf Ressourceneffizienz und Deployment-Flexibilität. Es soll geprüft werden, ob der Einsatz von Rust für dateiintensive Operationen (ZIP-Erstellung, I/O) in Kombination mit einem SvelteKit-Frontend messbare Vorteile gegenüber einer äquivalenten Node.js-basierten Architektur (Electron) bietet \cite{tauriDocs2025}.

\subsection{Forschungsfragen}
Zur Erreichung des Ziels werden folgende Forschungsfragen (FF) beantwortet:

\begin{itemize}
    \item \textbf{FF1 (Artefaktgröße \& Verteilbarkeit):} Wie verhalten sich die Dateigrößen der Installationsmedien von Tauri und Electron im Vergleich und wird die E-Mail-Versandfähigkeit durch Tauri gewährleistet?
    \item \textbf{FF2 (Ressourceneffizienz):} Welchen Einfluss hat die Auslagerung der Dateiverarbeitung (ZIP-Komprimierung, JSON-Generierung) in ein Rust-Backend auf die CPU-Last und den Arbeitsspeicherverbrauch im Vergleich zu einer Node.js-Laufzeitumgebung \cite{electronDocs}?
    \item \textbf{FF3 (Installationsprozess):} Inwiefern ermöglicht die NSIS-Integration in Tauri v2 eine flexiblere Anpassung des Setups (Silent Install, Non-Admin Mode, Custom UI) im Vergleich zu Standard-Electron-Buildern?
\end{itemize}

\section{Methodik und geplantes Vorgehen} \label{sec:methodik}

Die Arbeit folgt einem methodischen Mix aus \textbf{quantitativer Messung} und \textbf{qualitativer Analyse}. Dabei orientieren sich die Qualitätskriterien an der Norm ISO 25010 (Effizienz und Portabilität) \cite{iso25010}.

\subsection{Vergleichende Implementierung (Quantitative Methode)}
Die Kernlogik der GBI-Anwendung ist bereits in Tauri (Rust/SvelteKit) implementiert.
Für den wissenschaftlichen Vergleich wird ein \textbf{Referenz-Prototyp} in Electron erstellt.
Dieser Prototyp bildet die kritischen Pfade der Anwendung nach:
\begin{enumerate}
    \item Das Rendern der Formulare.
    \item Das Generieren der JSON-Strukturen und das Packen des ZIP-Archivs (in Electron mittels Node.js-Modulen realisiert).
\end{enumerate}
Anschließend werden auf einem Referenzsystem (Windows 10/11) standardisierte Messungen durchgeführt (Speicherverbrauch im Leerlauf, CPU-Peaks beim Zippen, finale Größe der `.exe`). Als Referenz für die Messmethodik dient die Studie von Thangadurai et al. \cite{thangaduraiElectronVsWeb2024}.

\subsection{Analyse der Deployment-Fähigkeiten (Qualitative Methode)}
Es wird eine Nutzwertanalyse der Installer-Technologien durchgeführt.
Dabei wird untersucht, wie granular sich der NSIS-Installer in Tauri konfigurieren lässt (z.B. Einbinden von Lizenzvereinbarungen, Bildmaterial, Installationspfad-Wahl ohne Admin-Rechte) und wie sich dies zum Konfigurationsaufwand in Electron verhält.
Zusätzlich wird der Update-Prozess (Signierung, Patch-Verteilung) evaluiert.

\section{Erwartete Resultate} \label{sec:erw_resultate}
Es wird erwartet, dass die Tauri-Anwendung eine Installer-Größe von unter 10 MB erreicht (vs. >60 MB bei Electron), was den E-Mail-Versand ermöglicht.
Durch die Nutzung von Rust für die ZIP-Komprimierung wird eine performantere Abarbeitung erwartet als im Node.js Main-Process von Electron. Bezüglich des Installers soll gezeigt werden, dass NSIS die Anforderungen an eine benutzerfreundliche Installation ohne Administratorrechte vollständig erfüllt.
