% nutzen Sie \chapter wenn Dokumentenklasse auf "book" ist
% ansonsten nutzen Sie \section bei articel und anderen
\section{Zielsetzung und Forschungsfragen} \label{sec:zielsetzung}

Ziel der Abschlussarbeit ist die technische Evaluierung von Tauri v2 für industrielle Desktop-Anwendungen mit Fokus auf Ressourceneffizienz und Deployment-Flexibilität. Es wird geprüft, ob der Einsatz von Rust für dateiintensive Operationen (ZIP-Erstellung, I/O) mit SvelteKit-Frontend messbare Vorteile gegenüber Node.js-basierter Architektur (Electron) bietet \cite{tauriDocs2025}.

\subsection{Forschungsfragen}
Zur Erreichung des Ziels werden folgende Forschungsfragen (FF) beantwortet:

\begin{itemize}
    \item \textbf{FF1 (Artefaktgröße \& Verteilbarkeit):} Wie verhalten sich die Dateigrößen der Installationsmedien von Tauri und Electron im Vergleich und wird die E-Mail-Versandfähigkeit durch Tauri gewährleistet?
    \item \textbf{FF2 (Ressourceneffizienz):} Welchen Einfluss hat die Auslagerung der Dateiverarbeitung (ZIP-Komprimierung, JSON-Generierung) in ein Rust-Backend auf die CPU-Last und den Arbeitsspeicherverbrauch im Vergleich zu einer Node.js-Laufzeitumgebung?
    \item \textbf{FF3 (Installationsprozess):} Inwiefern ermöglicht die NSIS-Integration in Tauri v2 eine flexiblere Anpassung des Setups (Silent Install, Non-Admin Mode, Custom UI) im Vergleich zu Standard-Electron-Buildern?
\end{itemize}

\subsection{Erwartete Resultate} \label{sec:erw_resultate}
Es wird erwartet, dass die Tauri-Anwendung eine Installer-Größe von unter 30 MB erreicht (vs. >60 MB bei Electron), was E-Mail-Versand ermöglicht.
Die Nutzung von Rust für ZIP-Komprimierung lässt eine performantere Abarbeitung im Vergleich zum Node.js Main-Process von Electron erwarten. NSIS soll eine benutzerfreundliche Installation ohne Admin-Rechte vollständig ermöglichen.
