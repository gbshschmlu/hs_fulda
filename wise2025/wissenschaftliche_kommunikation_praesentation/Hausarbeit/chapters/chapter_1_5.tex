% nutzen Sie \chapter wenn Dokumentenklasse auf "book" ist
% ansonsten nutzen Sie \section bei articel und anderen
\section{Stand der Forschung} \label{sec:stand-der-forschung}

Cross-Platform-Frameworks für Desktop-Anwendungen sind ein aktives Forschungsfeld.
Die folgende Darstellung ordnet die geplante Arbeit in den wissenschaftlichen Kontext ein.

\subsection{Electron als etablierter Standard}
Electron hat sich seit 2013 als dominierendes Framework für plattformübergreifende Desktop-Anwendungen etabliert.
Prominente Anwendungen wie Visual Studio Code, Slack oder Microsoft Teams nutzen diese Technologie \cite{electronDocs}.
Die Architektur kombiniert Chromium als Rendering-Engine mit Node.js für Backend-Funktionalität.

Thangadurai et al. \cite{thangaduraiElectronVsWeb2024} verglichen in ihrer Studie den Energie- und Ressourcenverbrauch von Electron-Anwendungen mit äquivalenten Web-Anwendungen.
Sie stellten fest, dass Electron-Anwendungen aufgrund der gebündelten Chromium-Engine typischerweise 80--120 MB Speicher im Leerlauf belegen und Installer-Größen von über 60 MB aufweisen.
Dies stellt insbesondere in Umgebungen mit limitierter Bandbreite oder älteren Systemen eine Herausforderung dar.

\subsection{Alternative Ansätze: Flutter, Qt und Tauri}
Parallel zu Electron haben sich alternative Frameworks entwickelt, die unterschiedliche Strategien zur Reduzierung des Ressourcenverbrauchs verfolgen:

\begin{itemize}
    \item \textbf{Flutter Desktop}: Googles UI-Framework nutzt die Dart-Sprache und rendert mit der eigenen Skia-Engine \cite{flutterArch}. Dies ermöglicht native Performance, erfordert jedoch das Erlernen einer neuen Programmiersprache.
    \item \textbf{Qt}: Als langjähriges natives Framework bietet Qt plattformübergreifende Entwicklung in C++ \cite{qtDocs}. Der Fokus liegt auf nativen Widgets, was zu kleineren Binaries führt, jedoch höhere Einstiegshürden mit sich bringt.
    \item \textbf{Tauri}: Das 2019 initiierte Framework verfolgt einen hybriden Ansatz, der Web-Technologien im Frontend mit Rust im Backend kombiniert \cite{tauriDocs2025}. Die zentrale Differenzierung zu Electron liegt in der Nutzung der betriebssystemeigenen Webview (WebView2 unter Windows, WebKit unter macOS/Linux) anstelle einer gebündelten Browser-Engine.
\end{itemize}

\subsection{Forschungslücke: Tauri v2 im Enterprise-Kontext}
Während die Architektur von Tauri in der technischen Dokumentation \cite{tauriDocs2025} beschrieben ist, existieren kaum wissenschaftliche Studien, die Tauri v2 – insbesondere mit erweiterten NSIS-Fähigkeiten – in einem industriellen B2B-Kontext evaluieren.
Die Arbeit von Thangadurai et al. \cite{thangaduraiElectronVsWeb2024} liefert Messmethodik für Performance-Vergleiche, bezieht Tauri jedoch nicht ein.

Die geplante Abschlussarbeit adressiert diese Lücke durch:
\begin{enumerate}
    \item Quantitative Evaluierung der Ressourceneffizienz von Tauri v2 im Vergleich zu Electron anhand einer realen Anwendung.
    \item Analyse der NSIS-Integration für Installationsszenarien ohne Administratorrechte.
    \item Überprüfung, ob die Vorteile von Rust (Speichersicherheit, Performance) sich in praxisrelevanten Anwendungsfällen (ZIP-Komprimierung, Dateiverwaltung) messbar niederschlagen.
\end{enumerate>

Damit leistet die Arbeit einen Beitrag zur Fundierung der Entscheidung zwischen modernen Cross-Platform-Frameworks in restriktiven Unternehmensumgebungen.

