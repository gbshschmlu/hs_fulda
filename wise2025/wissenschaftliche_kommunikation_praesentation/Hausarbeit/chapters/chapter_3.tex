% nutzen Sie \chapter wenn Dokumentenklasse auf "book" ist
% ansonsten nutzen Sie \section bei articel und anderen
\section{Methodik und geplantes Vorgehen} \label{sec:methodik}

Die Arbeit folgt einem methodischen Mix aus \textbf{quantitativer Messung} und \textbf{qualitativer Analyse}.
Die Qualitätskriterien orientieren sich an ISO/IEC 25010 (Effizienz und Portabilität) \cite{iso25010}.

\subsection{Vergleichende Implementierung (Quantitative Methode)}
Die Kernlogik der GBI-Anwendung ist bereits in Tauri (Rust/SvelteKit) implementiert.
Für den wissenschaftlichen Vergleich wird ein \textbf{Referenz-Prototyp} in Electron erstellt, der die kritischen Pfade abbildet:
\begin{enumerate}
    \item Rendern der Formulare.
    \item Generieren der JSON-Strukturen und Packen des ZIP-Archivs (mittels Node.js-Modulen).
\end{enumerate}
Anschließend werden auf einem Referenzsystem (Windows 10/11) standardisierte Messungen durchgeführt (Speicherverbrauch im Leerlauf, CPU-Peaks beim Zippen, finale \texttt{.exe}-Größe). Messmethodik nach Thangadurai et al. \cite{thangaduraiElectronVsWeb2024}.

\subsection{Nutzwertanalyse der Deployment-Fähigkeiten (Qualitative Methode)}
Die Installer-Technologien werden mittels Nutzwertanalyse bewertet.
Gewichtete Kriterien:

\begin{itemize}
    \item \textbf{Konfigurationsaufwand} (Gewichtung 25\%): Zeitaufwand und Komplexität für die initiale Konfiguration des Installers.
    \item \textbf{Anpassbarkeit} (Gewichtung 30\%): Möglichkeit zur Integration von Lizenztexten, Branding und benutzerdefinierten Dialogen.
    \item \textbf{Installation ohne Admin-Rechte} (Gewichtung 35\%): Funktionalität und Zuverlässigkeit der User-Mode-Installation.
    \item \textbf{Update-Mechanismus} (Gewichtung 10\%): Unterstützung für automatische Updates und Code-Signierung.
\end{itemize}

Die Gewichtungsfaktoren reflektieren die Prioritäten des B2B-Anwendungsfalls, wobei die Installation ohne Administratorrechte als kritischste Anforderung eingestuft wird.

\subsection{Evaluationskriterien}
Die Bewertung erfolgt auf einer 5-stufigen Skala (1 = ungenügend, 5 = exzellent).
Für jedes Framework wird der gewichtete Gesamtnutzen berechnet.

Die Bewertungsskala wird exemplarisch am Kriterium \textbf{Installation ohne Admin-Rechte} konkretisiert:
\begin{itemize}
    \item \textbf{5 (exzellent):} Installation funktioniert zuverlässig ohne Admin-Rechte, vollständig dokumentiert, keine manuellen Anpassungen erforderlich.
    \item \textbf{4 (gut):} Installation ohne Admin-Rechte möglich, geringe manuelle Nacharbeiten oder Konfigurationsschritte nötig.
    \item \textbf{3 (befriedigend):} Installation möglich, jedoch mit Einschränkungen (z.B. eingeschränkte Funktionalität, Workarounds erforderlich).
    \item \textbf{2 (ausreichend):} Installation technisch möglich, aber mit erheblichem Aufwand oder instabil.
    \item \textbf{1 (ungenügend):} Installation ohne Admin-Rechte nicht möglich oder nicht praktikabel umsetzbar.
\end{itemize}

Für die übrigen Kriterien wird analog verfahren: Höhere Punktzahlen stehen für geringeren Aufwand (\textit{Konfigurationsaufwand}), bessere Anpassungsmöglichkeiten (\textit{Anpassbarkeit}) bzw. umfassendere Update-Funktionen (\textit{Update-Mechanismus}).

% Neue Seite da sonst der Section-Titel eine Seite darüber steht und viel Platz verschwendet
\newpage
\section{Zeitplan und Ressourcen} \label{sec:zeitplan}

\begin{center}
    \begin{figure}[htbp]
    \centering
    \begin{ganttchart}[
        expand chart=\textwidth,
        vgrid, hgrid,
        y unit chart=0.7cm,
        bar height=0.4,
        bar top shift=0.3,
        title height=1,
        title label font=\bfseries\small,
        bar label font=\small,
        group label font=\small\bfseries,
        milestone label font=\small\itshape
    ]{1}{12}

        % Titelzeile
        \gantttitle{Zeitplan in Wochen (12 Wochen Gesamt)}{12} \\
        \gantttitlelist{1,...,12}{1} \\

        % --- Phase 1 ---
        \ganttgroup{1. Analyse \& Theorie}{1}{4} \\
        \ganttbar{Recherche \& Framework-Analyse}{1}{2} \\
        \ganttbar{Verfassen: Stand der Forschung}{2}{4} \\

        % --- Phase 2 ---
        \ganttgroup{2. Implementierung}{3}{8} \\
        \ganttbar{Electron-Prototyp (Baseline)}{3}{4} \\
        \ganttbar{Tauri \& NSIS Konfiguration}{5}{7} \\
        \ganttbar{Durchführung Messreihen}{8}{8} \\

        % --- Phase 3 ---
        \ganttgroup{3. Finalisierung}{9}{12} \\
        \ganttbar{Auswertung \& Diskussion}{9}{10} \\
        \ganttbar{Feinschliff \& Abstract}{10}{11} \\
        \ganttbar{Puffer \& Abgabe}{12}{12} \\

        \ganttmilestone{Abgabe}{12}

        % Verbindungen (Links) visualisieren Abhängigkeiten
        \ganttlink{elem1}{elem2} % Recherche -> Schreiben
        \ganttlink{elem3}{elem4} % Schreiben -> Electron Start
        \ganttlink{elem4}{elem5} % Electron -> Tauri
        \ganttlink{elem5}{elem6} % Tauri -> Messen
        \ganttlink{elem6}{elem7} % Messen -> Auswerten

    \end{ganttchart}
    \caption{Zeitplan der Bachelorarbeit: Paralleles Verfassen der theoretischen Grundlagen während der Implementierungsphase.}
    \label{fig:zeitplan}
\end{figure}

\end{center}
