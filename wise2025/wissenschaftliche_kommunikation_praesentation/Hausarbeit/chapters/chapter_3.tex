% nutzen Sie \chapter wenn Dokumentenklasse auf "book" ist
% ansonsten nutzen Sie \section bei articel und anderen
%\chapter{Kapitel} \label{kapitel}
\section{Methodik und geplantes Vorgehen} \label{sec:methodik}

Die Arbeit folgt einem methodischen Mix aus \textbf{quantitativer Messung} und \textbf{qualitativer Analyse}. Dabei orientieren sich die Qualitätskriterien an der Norm ISO/IEC 25010 (Effizienz und Portabilität) \cite{iso25010}.

\subsection{Vergleichende Implementierung (Quantitative Methode)}
Die Kernlogik der GBI-Anwendung ist bereits in Tauri (Rust/SvelteKit) implementiert.
Für den wissenschaftlichen Vergleich wird ein \textbf{Referenz-Prototyp} in Electron erstellt.
Dieser Prototyp bildet die kritischen Pfade der Anwendung nach:
\begin{enumerate}
    \item Das Rendern der Formulare.
    \item Das Generieren der JSON-Strukturen und das Packen des ZIP-Archivs (in Electron mittels Node.js-Modulen realisiert).
\end{enumerate}
Anschließend werden auf einem Referenzsystem (Windows 10/11) standardisierte Messungen durchgeführt (Speicherverbrauch im Leerlauf, CPU-Peaks beim Zippen, finale Größe der \texttt{.exe}). Als Referenz für die Messmethodik dient die Studie von Thangadurai et al. \cite{thangaduraiElectronVsWeb2024}.

\subsection{Nutzwertanalyse der Deployment-Fähigkeiten (Qualitative Methode)}
Es wird eine Nutzwertanalyse der Installer-Technologien durchgeführt.
Die Bewertung erfolgt anhand folgender gewichteter Kriterien:

\begin{itemize}
    \item \textbf{Konfigurationsaufwand} (Gewichtung 25\%): Zeitaufwand und Komplexität für die initiale Konfiguration des Installers.
    \item \textbf{Anpassbarkeit} (Gewichtung 30\%): Möglichkeit zur Integration von Lizenztexten, Branding und benutzerdefinierten Dialogen.
    \item \textbf{Installation ohne Admin-Rechte} (Gewichtung 35\%): Funktionalität und Zuverlässigkeit der User-Mode-Installation.
    \item \textbf{Update-Mechanismus} (Gewichtung 10\%): Unterstützung für automatische Updates und Code-Signierung.
\end{itemize}

Die Gewichtungsfaktoren reflektieren die Prioritäten des B2B-Anwendungsfalls, wobei die Installation ohne Administratorrechte als kritischste Anforderung eingestuft wird.

\subsection{Evaluationskriterien}
Die Bewertung erfolgt auf einer 5-stufigen Skala (1 = ungenügend, 5 = exzellent).
Für jedes Framework wird der gewichtete Gesamtnutzen berechnet und dokumentiert.
Zusätzlich wird der Update-Prozess (Signierung, Patch-Verteilung) evaluiert.

\section{Zeitplan und Ressourcen} \label{sec:zeitplan}

\begin{center}
    \begin{figure}[htbp]
    \centering
    \begin{ganttchart}[
        expand chart=\textwidth,
        vgrid, hgrid,
        y unit chart=0.7cm,
        bar height=0.4,
        bar top shift=0.3,
        title height=1,
        title label font=\bfseries\small,
        bar label font=\small,
        group label font=\small\bfseries,
        milestone label font=\small\itshape
    ]{1}{12}

        % Titelzeile
        \gantttitle{Zeitplan in Wochen (12 Wochen Gesamt)}{12} \\
        \gantttitlelist{1,...,12}{1} \\

        % --- Phase 1 ---
        \ganttgroup{1. Analyse \& Theorie}{1}{4} \\
        \ganttbar{Recherche \& Framework-Analyse}{1}{2} \\
        \ganttbar{Verfassen: Stand der Forschung}{2}{4} \\

        % --- Phase 2 ---
        \ganttgroup{2. Implementierung}{3}{8} \\
        \ganttbar{Electron-Prototyp (Baseline)}{3}{4} \\
        \ganttbar{Tauri \& NSIS Konfiguration}{5}{7} \\
        \ganttbar{Durchführung Messreihen}{8}{8} \\

        % --- Phase 3 ---
        \ganttgroup{3. Finalisierung}{9}{12} \\
        \ganttbar{Auswertung \& Diskussion}{9}{10} \\
        \ganttbar{Feinschliff \& Abstract}{10}{11} \\
        \ganttbar{Puffer \& Abgabe}{12}{12} \\

        \ganttmilestone{Abgabe}{12}

        % Verbindungen (Links) visualisieren Abhängigkeiten
        \ganttlink{elem1}{elem2} % Recherche -> Schreiben
        \ganttlink{elem3}{elem4} % Schreiben -> Electron Start
        \ganttlink{elem4}{elem5} % Electron -> Tauri
        \ganttlink{elem5}{elem6} % Tauri -> Messen
        \ganttlink{elem6}{elem7} % Messen -> Auswerten

    \end{ganttchart}
    \caption{Zeitplan der Bachelorarbeit: Paralleles Verfassen der theoretischen Grundlagen während der Implementierungsphase.}
    \label{fig:zeitplan}
\end{figure}

\end{center}
