\section{Verzeichnis der verwendeten Werkzeuge}
\label{sec:tools}

\begin{list}{}{%
    \setlength{\labelwidth}{2.0cm}%
    \setlength{\labelsep}{0.3cm}%
    \setlength{\leftmargin}{2.5cm}%
    \setlength{\itemindent}{0cm}%
    \setlength{\listparindent}{0cm}%
}

    \item[\textbf{[Gemini]}] 3.0 Pro Preview, Google\\
    \emph{Verwendung:}
    \begin{itemize}
        \item Brainstorming zur Themenfindung und Abgrenzung (Präsentation vs. Exposé)
        \item Formulierungshilfen in: Kap. \ref{sec:einleitung}, Abs. \ref{sec:motivation}, Kap. \ref{sec:zielsetzung} und Kap. \ref{sec:methodik}
        \item Erstellung eines Basis-Entwurfs für: Mermaid-Architekturdiagramm (Abb. \ref{fig:architektur}); anschließend manuell überarbeitet und gestylt.\\
        \footnotesize \textit{Prompt:} \texttt{ Erstelle ein Mermaid Graph Diagramm, das die Architektur von Electron (mit Node/Chromium) und Tauri (mit Rust/OS Webview) gegenüberstellt} \normalsize
        \item Literaturrecherche
    \end{itemize}

    \item[\textbf{[Claude]}] Sonnet 4.5, Anthropic\\
    \emph{Verwendung:}
    \begin{itemize}
        \item Formulierungshilfen in Kapitel \ref{sec:stand-der-forschung}
        \item Umstrukturierung der Methodik in Kap. \ref{sec:methodik}
        \item Überprüfung der wissenschaftlichen Sprache und Stilistik
        \item Reformulierung und Präzisierung in Kap. \ref{sec:einleitung}, \ref{sec:zielsetzung}, \ref{sec:methodik}
        \item Literaturrecherche
    \end{itemize}
\end{list}
