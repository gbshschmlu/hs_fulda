\subsection{Vorläufige Gliederung der Abschlussarbeit}

\begin{enumerate}
    \renewcommand{\labelenumii}{\arabic{enumi}.\arabic{enumii}}
    \renewcommand{\labelenumiii}{\arabic{enumi}.\arabic{enumii}.\arabic{enumiii}}

    \item \textbf{Einleitung}
    \begin{enumerate}
        \item Motivation und Ausgangslage bei der Grenzebach BSH GmbH
        \item Problemstellung: Herausforderungen bei der Softwareverteilung in heterogenen Lieferantennetzwerken
        \item Zielsetzung der Arbeit
        \item Forschungsfragen
        \item Aufbau der Arbeit
    \end{enumerate}

    \item \textbf{Theoretische Grundlagen und Stand der Technik}
    \begin{enumerate}
        \item Architekturmodelle für Cross-Platform-Desktop-Anwendungen
        \item Analyse des Frameworks Electron (Node.js und Chromium)
        \item Analyse des Frameworks Tauri v2 (Rust und System-Webview)
        \item Technologien zur Software-Installation (NSIS vs. WiX vs. MSI)
        \item Kriterien der Softwarequalität nach ISO 25010 (Fokus: Effizienz und Portabilität)
    \end{enumerate}

    \item \textbf{Konzeption der Vergleichsstudie}
    \begin{enumerate}
        \item Definition des Anwendungsfalls: Das GBI-Tool
        \item Anforderungsanalyse an den Rollout-Prozess (Silent Install, Non-Admin)
        \item Definition der Messmetriken (Artefaktgröße, RAM, CPU, Startzeit)
        \item Versuchsaufbau und Beschreibung der Testumgebung
    \end{enumerate}

    \item \textbf{Implementierung der Testumgebung}
    \begin{enumerate}
        \item Entwicklung eines Referenz-Prototypen in Electron (Baseline)
        \item Technische Umsetzung der GBI-Anwendung in Tauri
        \item Konfiguration des NSIS-Installers für restriktive Benutzerrechte
        \item Implementierung des Update-Mechanismus
    \end{enumerate}

    \item \textbf{Evaluation und Ergebnisse}
    \begin{enumerate}
        \item Vergleich der Installer- und Anwendungsgrößen
        \item Messergebnisse zur Laufzeitperformance (Speicher und CPU)
        \item Qualitative Bewertung des Deployment-Prozesses und der Anpassbarkeit
        \item Analyse der Kompatibilität auf verschiedenen Windows-Versionen
    \end{enumerate}

    \item \textbf{Diskussion}
    \begin{enumerate}
        \item Interpretation der Messergebnisse im industriellen Kontext
        \item Abwägung: Entwickler-Experience (Rust) vs. Performance-Gewinn
        \item Risikobetrachtung: Abhängigkeit von der Webview2-Runtime
        \item Handlungsempfehlung für die Grenzebach BSH GmbH
    \end{enumerate}

    \item \textbf{Fazit und Ausblick}
    \begin{enumerate}
        \item Zusammenfassung der Ergebnisse
        \item Ausblick: Portierung auf weitere Plattformen (macOS/Linux)
    \end{enumerate}
\end{enumerate}
