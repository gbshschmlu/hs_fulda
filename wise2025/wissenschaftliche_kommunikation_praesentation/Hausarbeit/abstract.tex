%%%%%%%%%%%%%%%%%%%%%%%%%%%%%%%%%%%%%%%%%%%%%%%%%%
%%%%		~~~~ Abstrakt ~~~~
%%%%%%%%%%%%%%%%%%%%%%%%%%%%%%%%%%%%%%%%%%%%%%%%%%
% Sollte wirklich kurz sein, 300 Wörter, maximal 500
\begin{abstract}
    \justifying
    Die Digitalisierung von Lieferketten erfordert Softwarelösungen, die in heterogenen IT-Umgebungen funktionieren.
    Bei der Entwicklung einer Desktop-Anwendung zur Erfassung und Übermittlung von Produktspezifikationen (GBI - Grenzebach BOM Importer) müssen Unternehmen Software an Lieferanten mit unterschiedlicher Hardwareausstattung und restriktiven IT-Richtlinien verteilen. Frameworks wie Electron bündeln eine vollständige Browser-Engine, was zu Dateigrößen von über 60 MB führt. Dies erschwert sowohl die Verteilung per E-Mail als auch die Installation ohne Admin-Rechte.

    Dieses Exposé skizziert eine vergleichende Analyse von Tauri v2 und Electron.
    Die Evaluierung untersucht, ob Tauri durch die Verwendung systemseitiger Webviews und eines Rust-Backends die Anforderungen an minimale Artefaktgrößen und ressourcenschonenden Betrieb besser erfüllt als Electron.
    Ein besonderer Fokus liegt auf der Anpassung des Installationsprozesses mittels NSIS für Umgebungen ohne Admin-Rechte.
    Durch quantitative Messungen und eine Nutzwertanalyse soll die Arbeit eine fundierte Entscheidungsgrundlage für die Auswahl moderner Cross-Platform-Technologien in B2B-Szenarien schaffen.
\end{abstract}
