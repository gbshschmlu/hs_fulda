%%%%%%%%%%%%%%%%%%%%%%%%%%%%%%%%%%%%%%%%%%%%%%%%%%
%%%%		~~~~ Abstrakt ~~~~
%%%%%%%%%%%%%%%%%%%%%%%%%%%%%%%%%%%%%%%%%%%%%%%%%%
% Sollte wirklich kurz sein, 300 Wörter, maximal 500
\begin{abstract}
    \justifying
    Die Digitalisierung von Lieferketten erfordert Softwarelösungen für heterogene IT-Landschaften.
    Bei der Entwicklung einer Desktop-Anwendung zur Erfassung und Übermittlung von Produktspezifikationen (GBI) müssen Unternehmen Software an Lieferanten mit unbekannter Hardwareausstattung und restriktiven IT-Richtlinien verteilen. Frameworks wie Electron bündeln eine vollständige Browser-Engine, was zu Dateigrößen von über 60 MB führt und die E-Mail-Verteilung sowie Installation ohne Admin-Rechte erschwert.

    Dieses Exposé skizziert eine vergleichende Analyse von Tauri v2 und Electron.
    Die Evaluierung prüft, ob Tauri durch systemseitige Webviews und ein Rust-Backend die Anforderungen an minimale Artefaktgrößen und ressourcenschonenden Betrieb besser erfüllt.
    Ein Schwerpunkt liegt auf der Anpassbarkeit des Installationsprozesses mittels NSIS für Umgebungen ohne Admin-Rechte.
    Die Arbeit liefert durch quantitative Messungen und Nutzwertanalyse eine fundierte Entscheidungsgrundlage für moderne Cross-Platform-Technologien in B2B-Umgebungen.
\end{abstract}
