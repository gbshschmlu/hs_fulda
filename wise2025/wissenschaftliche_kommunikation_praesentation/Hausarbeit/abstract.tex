%%%%%%%%%%%%%%%%%%%%%%%%%%%%%%%%%%%%%%%%%%%%%%%%%%
%%%%		~~~~ Abstrakt ~~~~
%%%%%%%%%%%%%%%%%%%%%%%%%%%%%%%%%%%%%%%%%%%%%%%%%%
% Sollte wirklich kurz sein, 300 Wörter, maximal 500
\begin{abstract}
    \justifying
    Die Digitalisierung von Lieferketten erfordert Softwarelösungen, die sich nahtlos in heterogene IT-Landschaften integrieren lassen.
    Bei der Entwicklung einer Desktop-Anwendung zur Erfassung und Übermittlung von Produktspezifikationen (GBI) stehen Unternehmen vor der Herausforderung, Software an Lieferanten mit unbekannter Hardwareausstattung und restriktiven IT-Richtlinien zu verteilen. Etablierte Frameworks wie Electron bündeln eine vollständige Browser-Laufzeitumgebung, was zu hohen Dateigrößen führt und die Verteilung via E-Mail sowie die Installation ohne Administratorrechte erschwert.

    Dieses Exposé skizziert das Vorhaben einer vergleichenden Analyse des Frameworks Tauri v2 gegenüber Electron.
    Ziel ist die Evaluierung, ob Tauri durch die Nutzung systemseitiger Webviews und eines Rust-basierten Backends die Anforderungen an minimale Artefaktgrößen und ressourcenschonenden Betrieb besser erfüllen kann.
    Ein besonderer Fokus wird auf der Anpassbarkeit des Installationsprozesses mittels NSIS (Nullsoft Scriptable Install System) für Umgebungen ohne privilegierte Nutzerrechte liegen.
    Die geplante Arbeit soll durch quantitative Messungen und eine Nutzwertanalyse eine fundierte Entscheidungsgrundlage für den Einsatz moderner Cross-Platform-Technologien in restriktiven B2B-Umfeldern liefern.
\end{abstract}
