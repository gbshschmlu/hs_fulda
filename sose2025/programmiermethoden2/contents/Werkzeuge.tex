\section{Verzeichnis der verwendeten Werkzeuge}
\label{sec:tools}

% Eigene Umgebung, die wie thebibliography aussieht, aber keine eigene Überschrift hat
\begin{list}{}{%
\setlength{\labelwidth}{1.5cm}%
\setlength{\labelsep}{0.3cm}%
\setlength{\leftmargin}{2cm}%
\setlength{\itemindent}{0cm}%
\setlength{\listparindent}{0cm}%
}

\item[\textbf{[Claude]}] \textbf{Claude} (3.7 Sonnet, Anthropic)\\
\emph{Verwendung:} 
\begin{itemize}
  \item Beihilfe für die Gliederung und Strukturierung der Hausarbeit
  \item Überarbeitung von Formulierungen in: \ref{sec:einleitung}, \ref{sec:grundlagen}, \ref{sec:evaluation_versionskontrolle}
\end{itemize}

\item[\textbf{[Claude]}] \textbf{Claude} (4 Opus, Anthropic)\\
\emph{Verwendung:} 
\begin{itemize}
  \item Rechtschreibprüfung Gesamtdokument
  \item Konsistenzprüfung Gesamtdokument
  \item Quellenprüfung Gesamtdokument
  \item  Korrekturempfehlungen Gesamtdokument
\end{itemize}

\item[\textbf{[Gemini]}] \textbf{Gemini} (Gemini 2.5 Pro, Google)\\
\emph{Verwendung:} 
\begin{itemize}
  \item Beihilfe der Strukturierung von \ref{sec:einleitung}, \ref{sec:optimierung}
  \item Rechtschreibprüfung in: \ref{sec:einleitung}, \ref{sec:grundlagen}, \ref{sec:analyse_dokumentation}, \ref{sec:evaluation_versionskontrolle}, \ref{sec:optimierung}, \ref{sec:fazit}
  \item Literaturempfehlungen
\end{itemize}

\end{list}